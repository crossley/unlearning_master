\section*{Discussion}
\subsection*{Summary}
Feedback contingency --- defined here as the degree of randomness between
response and outcome --- may be key to flexibly modifying procedural memories
\cite{crossley_erasing_2013}. To our knowledge, this paper reports the first
behavioral experiments designed to shed light the neural implementation of
feedback contingency computation. We attempted to identify prefrontal network
contributions by examining the effect of increased cognitive load on savings in
relearning in a procedural category learning task. If the computation of
feedback contingency depends on prefrontal networks (i.e., shares resources with
the numerical Stroop), then we predicted (1) slowed change in classification
performance during intervention, and (2) decreased savings in relearning
relative to a no dual-task control. Our results were consistent with each of
these predictions.

\subsection*{Weakness of Raw Savings}
Perhaps the most conspicuous and unanticipated feature of our data is the lack
of large and robust savings in any condition. Even in the no dual-task control,
savings is only clearly expressed on early blocks of reacquisition. We speculate
that this is due to the duration of the intervention phase, which was a full 100
trials longer than we have used in our previous research
\cite{crossley_erasing_2013, crossley_renewal_2014}. Does this finding suggest
that increased intervention duration is a possible key to true unlearning? While
possible, we feel that this possibility is doubtful. Our reasoning is that (1)
savings in the no dual-task control is clearly expressed during the first 50
trials of reacquisition, and (2) this initial savings is reduced to interference
by the end of the experiment. We suggest that this very plausibly reflects
participant fatigue, and therefore we are hesitant to strongly interpret this
finding. We plan to robustly examine this issue in future research.

\subsection*{Category Learning as a Procedural Skill}
A natural concern for readers unfamiliar with the category learning literature
is that our behavioral paradigm is a poor choice for studying procedural
behaviors. This concern is grounded in the very understandable intuition that a
task with such simple motor demands (e.g., push a button) cannot possibly task
procedural networks that are strongly tied to motor processes. However,
following theories that humans have multiple memory systems
\cite{EichenbaumCohen2001, Squire2004, Tulving2000}, a large database of
evidence suggests that humans also have multiple, qualitatively distinct
category-learning systems \cite{AshbyCOVIS1998, EricksonKruschke1998}. According
to this view, procedural memory is used to form many-to-one stimulus-to-response
mappings (S-R associations), whereas declarative memory is used to apply rules
and test explicit hypotheses about category membership.

The majority of this evidence comes from prior research with rule-based (RB) and
information-integration (II) category-learning tasks \cite{HelieRoederAshby2010,
NomuraEtAl2007, SotoEtAl2013, WaldschmidtAshby2011}. In RB tasks, the categories
can be learned via an explicit hypothesis-testing procedure
\cite{AshbyCOVIS1998}. In the simplest variant, only one dimension is relevant
(e.g., bar width), and the task is to discover this dimension and then map the
different dimensional values to the relevant categories. In II tasks, accuracy
is maximized only if information from two or more stimulus dimensions is
integrated perceptually at a pre-decisional stage \cite{AshbyGott1988}. In most
cases, the optimal strategy in II tasks is difficult or impossible to describe
verbally \cite{AshbyCOVIS1998}. Verbal rules may be (and sometimes are) applied,
but they lead to suboptimal performance.

At least 25 different behavioral dissociations tie II learning to procedural
memory and RB learning to declarative memory (for reviews, see
\cite{AshbyMaddox2005,AshbyMaddox2010,AshbyValentin2016}). This hypothesis is
further supported by a variety of investigations into the neural underpinnings
of successful II and RB learning. Specifically, success in RB tasks depends on a
broad neural network that includes the prefrontal cortex (PFC), anterior
cingulate, the head of the caudate nucleus, and medial temporal lobe
structures---regions that are also frequently associated with declarative memory
and executive attention \cite{BrownMarsden1988, FiloteoEtAl2007,
MuhammadWallisMiller2006, SegerCincotta2006}. Arguably, the most important
region in this network is the PFC, where rules are thought to be initially
represented \cite{MillerCohen2001, WallisAndersonMiller2001}. Success in II
tasks, on the other hand, depends on regions that have been implicated in
procedural memory, including the striatum, premotor cortex, and the associated
sensorimotor basal ganglia loop \cite{AshbyEnnis2006,
FiloteoMaddoxSalmonSong2005, KnowltonMangelsSquire1996, NomuraEtAl2007}. This
network is consistent with the idea that S-R associations are built at
cortical-striatal synapses via dopamine-dependent reinforcement learning
\cite{AshbyCrossley2011, HoukAdamsBarto1995, JoelNivRuppin2002}.

\subsection*{Therapeutic Relevance}
The old adage of ``it's like riding a bike'' is a surprisingly accurate
description of procedural knowledge, reflecting its remarkable retention over
years without practice. Paradigms designed to study procedural learning in the
lab have echoed this adage, reporting savings in learning up to a year after
training \cite{Romano2010,turner_long-term_2012}. However, the stability of
procedural memory comes at the cost of remarkable inflexibility. For example,
changing any stimulus or response parameter that was present during training can
prove catastrophic to performance \cite{Rozanov_2010,Dienes_1997}. While
resilience and inflexibility are desirable traits when a useful skill been
sufficiently learned, they can also lead to persistent maladaptive behaviors
that have serious negative consequences, and in some cases may prove detrimental
to a person's health (e.g., drug abuse). Unfortunately, neither the potential
for modification of procedural knowledge, nor a method to do so, are well
understood.

Our previous research identified the interplay between the striatal cholinergic
interneurons and the midbrain dopamine system in controlling the eligibility of
procedural knowledge for modification. Directly targeting this network for
improved interventions is unfortunately challenging, due to the difficulty of
manipulating and measuring subcortical netoworks. Here, we look for more easily
accessible cortical substrates that may control the striatal mechanism. Our
results indicate the prefrontal networks likely play an important role in
controlling the estimation of feedback contingency, and thereby may provide an
accessible cortical target for electrical or magnetic intervention.

 
% TODD SAYS: I really like the therapeutic relevance section. I worry a little
% that the second and third sections seem defensive. We did not get a robust
% savings throughout but we did get early savings and more importantly the
% comparisons between the control and dual task conditions show the predicted
% pattern. I guess I would argue for emphasizing and discussing that more than the
% partial lack of savings. Unfortunately, we still have to work to convince folks
% that II is procedural. You do a good job of dealing with this.
