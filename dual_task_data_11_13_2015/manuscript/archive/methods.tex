\section*{Methods}
\subsection*{Design}
There were four dual-task conditions (Condition 1 through 4) and one no
dual-task control condition (Condition 5). The dual-task conditions differed on
two dimensions, (1) the number of trials on which the dual-task was applied, and
(2) whether or not the onset of the dual-task preceded the onset of
intervention.

\subsection*{Participants}
163 participants were recruited from the University of Texas at Austin
undergraduate population. There were 30 participants in Condition 1, 34
participants in Condition 2, 32 participants in Condition 3, 33 participants in
Condition 4, and 34 participants in Condition 5. All participants completed the
study and received course credit for their participation. All participants had
normal or corrected-to-normal vision.

\subsection*{Stimuli and Categories}
Stimuli were black lines that varied across trials only in length (pixels) and
orientation (degrees counterclockwise rotation from horizontal). The stimuli are
illustrated graphically in Figure 1, and were identical to those used by
Crossley et al. (2013).

\subsection{Procedure}
Participants in all conditions were told that they were to categorize lines on
the basis of their length and orientation, that there were four equally-likely
categories, and that high levels of accuracy could be achieved. The experiment
included three phases: acquisition (300 trials), intervention (400 trials), and
reacquisition (150 trials). During acquisition and reacquisition, feedback was
based on the participant's response, whereas feedback was random during the
intervention. Participants were given no prior instructions about the phases,
and the transition from one phase to another occurred without any warning to the
participant.

At the start of each non-Stroop [TM1]trial, a fixation point was displayed for 1
second and then the stimulus appeared. The stimulus remained on the screen until
the participant generated a response by pressing the ``Z'' key for category
``A'', the ``W'' key for category B, the ``/'' key for category C, or the ``P''
key for category D. Written instructions informed participants of the category
label to button mappings. An ``invalid key'' message was displayed if any other
button was pressed. The word ``Correct'' was presented for 1 second if the
response was correct or the word ``Wrong'' was presented for 1 second if the
response was incorrect (except during the intervention phase in which feedback
was completely random).

Stroop trials began with a fixation point that was displayed for 1 second. The
category stimulus and the Stroop stimuli (numbers flanking the category
stimulus) were displayed simultaneously. After 200 ms the Stroop stimuli were
replaced by white rectangles which remained on the screen until they made a
category response. Responses emitted before the Stroop stimuli were replaced by
white rectangles were not accepted. Feedback about the category response was
given immediately in the same fashion as on non-Stroop trials. The word ``value''
or ``size'' then appeared on the screen prompting participants to indicate which
side contained the numerically larger or the physically larger number.
Participants pressed the ``F'' key to choose the number on the left or the ``J''
key to choose the number on the right. The word ``Correct'' was then again
presented for 1 second if the response to the Stroop task was correct or the
word ``Wrong'' was presented for 1 second if the response was incorrect. See
Figure 1 for example trials both including and excluding the Stroop component.
The Stroop task was included on trials 251-350 in condition 1, 251-450 in
condition 2, 251-550 in condition 3 and 400-650 in condition 4.

Participants were instructed to try their hardest on both task components but to
prioritize performance on the Stroop task. Both the category learning task and
the Stroop task were explained to participants prior to beginning the
experiment, and on screen messages warned them when the Stroop component would
begin, and again when it would end. These messages read, ``You will now perform
both the categorization task and the paired numbers task simultaneously. Keep
trying your hardest!'' and ``You have now finished the section with the paired
numbers task. You will now be shown only the line categorization task. Keep
trying your hardest.'' 85\% of Stroop trials the numerically larger number was
physically smaller. The proportion of Stroop trials that prompted ``size'' or
``value'' was split 50/50. Accuracy on the numerical Stroop task was indicated
at the top of the screen when they received feedback regarding their performance
on the concurrent task on each trial. This score was displayed in green if it
was above 80\% and red if it was below 80\%. Note that when we refer to the
``dual-task'', we are referring to the Stroop task just described. 