Individuals in drug rehabilitation often experience relapse when returned to the
original context of their drug use. Why do treatments generalize so poorly
across different contexts? We recently addressed this question in the domain of
procedural learning, which is thought to play an important role in forms of
addiction, bad habits, and other maladaptive states. Our work suggests that
feedback contingency, defined as the degree of randomness between response and
outcome, plays a vital role in controlling a gate that normally prevents
procedural knowledge from being modified during interventions. Here, we ask
whether or not feedback contingency is computed via declarative mechanisms
(e.g., prefrontal networks involved in working memory and executive reasoning).
Our rationale is as follows: If feedback contingency is computed by declarative
mechanisms, then increasing cognitive load during intervention via the
concurrent performance of an additional task should disrupt the accurate
estimation of contingency, and thereby prevent the gate on procedural learning
from closing. We report the results from an experiment following this logic that
suggest that feedback contingency is indeed supported by declarative systems.
