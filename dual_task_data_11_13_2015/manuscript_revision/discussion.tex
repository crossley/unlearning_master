\section*{Discussion}
\subsection*{Summary}
Feedback contingency, which we define as the correlation between response
confidence and feedback valence, can be manipulated experimentally by varying
the randomness of feedback. Our current and previous results
\cite{crossley_erasing_2013} suggest that such manipulations may be key to
flexibly modifying procedural memories. To our knowledge, this article reports
results from the first behavioral experiments that investigate the cognitive and
neural mechanisms that estimate feedback contingency. Specifically, our goal was
to determine whether prefrontal-based declarative memory mechanisms mediate
contingency estimation. If they do, then a dual task that depends on working
memory and executive function should make it more difficult for participants to
recognize the sudden onset of random feedback. In our experiments, behavioral
signatures of this difficulty would include (1) a slowed decrease in
classification accuracy during intervention, and (2) decreased savings in
relearning relative to a no dual-task control. Our results were consistent with
both of these predictions.

\subsection*{Weakness of Raw Savings}
Perhaps our most conspicuous and unanticipated result was the lack of large and
robust savings in any condition. Even in the no dual-task control, savings is
only clearly expressed during early blocks of reacquisition. We speculate that
this is due to the duration of the intervention phase, which was a full 100
trials longer than we have used in previous research
\cite{crossley_erasing_2013, crossley_renewal_2014}. Does this finding suggest
that longer interventions are a possible key to true unlearning? Several
features of our data argue against this hypothesis. First, participants in the
no dual-task control condition showed considerable savings during the first 50
trials (i.e., 2 blocks) of reacquisition (see Figure
\ref{fig:savings_per_block}). If the longer intervention caused unlearning then
such savings should not have occurred. Second, this initial savings is reversed
to an interference during the last half of the reacquisition phase. True
unlearning predicts zero savings throughout reacquisition. The most parsimonious
account of the negative savings that occurred in every condition during the
latter half of the reacquisition phase may be participant fatigue. We plan to
robustly examine this issue in future research.

If participant fatigue is at play at the end of the experiment, then is it
possible that the difference in savings we observed between the dual-task
conditions and the no dual-task control could be driven by fatigue as well? 
This possibility does not survive a close inspection at our data. First, if
participant fatigue is the driving factor, then there should be differences
between the dual-task conditions, which we did not observe. Second, we performed
a linear regression with overall experiment time --- a proxy for fatigue --- as
the predictor variable, and found that it did not provide a compelling account
of our data.

% Matt: I'm a little worried that we're setting ourselves up here for criticism.
% Couldn't a reviewer just extend this argument and say something like the
% following: ``Categorization with a dual task is more tiring than categorization
% without a dual task. Maybe the only thing the dual task did was fatigue subjects
% and that the effect of this was to essentially just shift the savings vs blocks
% plots to the right (e.g., note that if you shift the dual task plots 2 blocks to
% the right they largely overlap the no dual-task control curve).

\subsection*{Category Learning as a Procedural Skill}
A natural question for readers unfamiliar with the category-learning literature
is whether our behavioral paradigm is a good choice for studying procedural
behaviors. In other words, how can a task with such simple motor demands (e.g.,
push a button) possibly recruit procedural networks that are strongly tied to
motor processes? In fact, the empirical evidence is strong that performance
improvements in the classification task used here are mediated via procedural
learning and memory. A large database of evidence suggests that humans have
multiple, qualitatively distinct category-learning systems \cite{AshbyCOVIS1998,
AshbyMaddox2005, EricksonKruschke1998}, and according to this view, procedural
memory is used to form many-to-one stimulus-to-response mappings (S-R
associations), whereas declarative memory is used to apply rules and test
explicit hypotheses about category membership.

The majority of this evidence comes from prior research with rule-based (RB) and
information-integration (II) category-learning tasks \cite{HelieRoederAshby2010,
NomuraEtAl2007, SotoEtAl2013, WaldschmidtAshby2011}. In RB tasks, the categories
can be learned via an explicit hypothesis-testing procedure
\cite{AshbyCOVIS1998}. In the simplest variant, only one dimension is relevant
(e.g., bar width), and the task is to discover this dimension and then map the
different dimensional values to the relevant categories. In II tasks, accuracy
is maximized only if information from two or more stimulus dimensions is
integrated perceptually at a pre-decisional stage \cite{AshbyGott1988}. In most
cases, the optimal strategy in II tasks is difficult or impossible to describe
verbally \cite{AshbyCOVIS1998}. Verbal rules may be (and sometimes are) applied,
but they lead to suboptimal performance. The task used here (and illustrated in
Figure 1) was an II category-learning task.

At least 25 different behavioral dissociations tie II learning to procedural
memory (and RB learning to declarative memory; for reviews, see
\citeNP{AshbyMaddox2005,AshbyMaddox2010,AshbyValentin2016}). For example, one
behavioral signature of procedural learning is that because of its motor
component, switching the locations of the response keys interferes with
performance \cite{WillinghamEtAl2000}. In agreement with this result, switching
the locations of the response keys interferes with II categorization
performance, even when the task only includes two categories\footnote{In
contrast, the same button switch does not interfere with RB performance in tasks
where the RB categories are created by simply rotating the II categories by
$45^\circ$} \cite{AshbyEllWaldron2003, MaddoxBohilIng2004, SpieringAshby2008}.

This hypothesis is further supported by a variety of investigations into the
neural underpinnings of successful II and RB learning. Specifically, success in
RB tasks depends on a broad neural network that includes the prefrontal cortex
(PFC), anterior cingulate, the head of the caudate nucleus, and medial temporal
lobe structures---regions that are also frequently associated with declarative
memory and executive attention \cite{BrownMarsden1988, FiloteoEtAl2007,
MuhammadWallisMiller2006, SegerCincotta2006}. Success in II tasks, on the other
hand, depends on regions that have been implicated in procedural memory,
including the striatum, premotor cortex, and the associated sensorimotor basal
ganglia loop \cite{AshbyEnnis2006, FiloteoMaddoxSalmonSong2005,
KnowltonMangelsSquire1996, NomuraEtAl2007}. This network is consistent with the
idea that S-R associations are built at cortical-striatal synapses via
dopamine-dependent reinforcement learning \cite{AshbyCrossley2011,
HoukAdamsBarto1995, JoelNivRuppin2002}.

\subsection*{Therapeutic Relevance}
The old adage of ``it's like riding a bike'' is a surprisingly accurate
description of procedural knowledge, reflecting its remarkable retention over
years without practice. Paradigms designed to study procedural learning in the
lab have echoed this adage, reporting savings in learning up to a year after
training \cite{Romano2010,turner_long-term_2012}. However, the stability of
procedural memory comes at the cost of remarkable inflexibility. For example,
changing any stimulus or response parameter that was present during training can
prove catastrophic to performance \cite{Rozanov_2010,Dienes_1997}. While
resilience and inflexibility are desirable traits when a useful skill has been
sufficiently learned, they can also lead to persistent maladaptive behaviors
that have serious negative consequences, and in some cases may prove detrimental
to a person's health (e.g., drug abuse). Unfortunately, neither the potential
for modification of procedural knowledge, nor a method to do so, are well
understood.

Our previous research identified the interplay between the striatal cholinergic
interneurons and the midbrain dopamine system in controlling the eligibility of
procedural knowledge for modification. Directly targeting this network for
improved interventions is unfortunately challenging, due to the difficulty of
manipulating and measuring subcortical networks. Here, we look for more easily
accessible cortical substrates that may control the striatal mechanism. Our
results indicate that prefrontal networks likely play an important role in
controlling the estimation of feedback contingency, and therefore may provide an
accessible cortical target for electrical or magnetic intervention.